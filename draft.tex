%%
%% This is file `sample-sigconf-authordraft.tex',
%% generated with the docstrip utility.
%%
%% The original source files were:
%%
%% samples.dtx  (with options: `all,proceedings,bibtex,authordraft')
%% 
%% IMPORTANT NOTICE:
%% 
%% For the copyright see the source file.
%% 
%% Any modified versions of this file must be renamed
%% with new filenames distinct from sample-sigconf-authordraft.tex.
%% 
%% For distribution of the original source see the terms
%% for copying and modification in the file samples.dtx.
%% 
%% This generated file may be distributed as long as the
%% original source files, as listed above, are part of the
%% same distribution. (The sources need not necessarily be
%% in the same archive or directory.)
%%
%%
%% Commands for TeXCount
%TC:macro \cite [option:text,text]
%TC:macro \citep [option:text,text]
%TC:macro \citet [option:text,text]
%TC:envir table 0 1
%TC:envir table* 0 1
%TC:envir tabular [ignore] word
%TC:envir displaymath 0 word
%TC:envir math 0 word
%TC:envir comment 0 0
%%
%% The first command in your LaTeX source must be the \documentclass
%% command.
%%
%% For submission and review of your manuscript please change the
%% command to \documentclass[manuscript, screen, review]{acmart}.
%%
%% When submitting camera ready or to TAPS, please change the command
%% to \documentclass[sigconf]{acmart} or whichever template is required
%% for your publication.
%%
%%
\documentclass[sigconf]{acmart}
%%
%% \BibTeX command to typeset BibTeX logo in the docs
\AtBeginDocument{%
  \providecommand\BibTeX{{%
    Bib\TeX}}}

%% Rights management information.  This information is sent to you
%% when you complete the rights form.  These commands have SAMPLE
%% values in them; it is your responsibility as an author to replace
%% the commands and values with those provided to you when you
%% complete the rights form.


%%
%% Submission ID.
%% Use this when submitting an article to a sponsored event. You'll
%% receive a unique submission ID from the organizers
%% of the event, and this ID should be used as the parameter to this command.
%%\acmSubmissionID{123-A56-BU3}

%%
%% For managing citations, it is recommended to use bibliography
%% files in BibTeX format.
%%
%% You can then either use BibTeX with the ACM-Reference-Format style,
%% or BibLaTeX with the acmnumeric or acmauthoryear sytles, that include
%% support for advanced citation of software artefact from the
%% biblatex-software package, also separately available on CTAN.
%%
%% Look at the sample-*-biblatex.tex files for templates showcasing
%% the biblatex styles.
%%

%%
%% The majority of ACM publications use numbered citations and
%% references.  The command \citestyle{authoryear} switches to the
%% "author year" style.
%%
%% If you are preparing content for an event
%% sponsored by ACM SIGGRAPH, you must use the "author year" style of
%% citations and references.
%% Uncommenting
%% the next command will enable that style.
%%\citestyle{acmauthoryear}


%%
%% end of the preamble, start of the body of the document source.
\begin{document}

%%
%% The "title" command has an optional parameter,
%% allowing the author to define a "short title" to be used in page headers.
\title{The Name of the Title Is Hope}

%%
%% The "author" command and its associated commands are used to define
%% the authors and their affiliations.
%% Of note is the shared affiliation of the first two authors, and the
%% "authornote" and "authornotemark" commands
%% used to denote shared contribution to the research.
\author{Qiyu Feng}
\email{tomato_fish@sjtu.edu.cn}
\affiliation{%
  \institution{Shanghai Jiao Tong University}
  \city{Shanghai}
  \country{China}
}

%%
%% By default, the full list of authors will be used in the page
%% headers. Often, this list is too long, and will overlap
%% other information printed in the page headers. This command allows
%% the author to define a more concise list
%% of authors' names for this purpose.
% \renewcommand{\shortauthors}{Trovato et al.}

\received{20 February 2007}
\received[revised]{12 March 2009}
\received[accepted]{5 June 2009}

%%
%% This command processes the author and affiliation and title
%% information and builds the first part of the formatted document.
\maketitle

\section{Introduction}
ACM's consolidated article template, introduced in 2017, provides a
consistent \LaTeX\ style for use across ACM publications, and
incorporates accessibility and metadata-extraction functionality
necessary for future Digital Library endeavors. Numerous ACM and
SIG-specific \LaTeX\ templates have been examined, and their unique
features incorporated into this single new template.

If you are new to publishing with ACM, this document is a valuable
guide to the process of preparing your work for publication. If you
have published with ACM before, this document provides insight and
instruction into more recent changes to the article template.

The ``\verb|acmart|'' document class can be used to prepare articles
for any ACM publication --- conference or journal, and for any stage
of publication, from review to final ``camera-ready'' copy, to the
author's own version, with {\itshape very} few changes to the source.

\section{Anything}

The most significant difference between quantum and classical gates lies in their reversibility: quantum gates are reversible, as all operators in pure quantum computation are unitary. Consequently, a fundamental question arises: Can quantum gates perform any operation that classical gates can?

The core problem is that certain logical operations are irreversible. For instance, an AND gate maps inputs $a$ and $b$ to $a \land b$; however, it is impossible to reconstruct $a$ and $b$ solely from $a \land b$. This is because the output contains less information than the input. To address this, a practical method is to add more output. For example, to compute $f(a,b) = a+b$, the function can be transformed into $f'(a,b)=(a, a+b)$. Consequently, the original inputs can be retrieved from the inverse operation $f'^{-1}(a, a+b)=(a,b)$. 

The situation becomes more complex, however, when considering the AND gate. If one attempts to define a mapping such that $AND(a, b) = (a \land b, c)$, the operation fails to be injective. Specifically, when $a \land b = 0$, there are three possible input pairs $(a, b)$, making it impossible for the single qubit $c$ to uniquely identify the original input. Literature suggests that an auxiliary qubit is required for assistance. Consider a transformation $AND(a, b, c) = (d, e, f)$; simply setting an output such as $d = a \land b$ does not resolve the issue. The ingenuity of this method, however, lies in restricting the domain of the auxiliary input qubit $c$. By constraining the viable domain $\{(a, b, c)\}$ to a specific subset of $\{0, 1\}^3$, the operation is not required to act as "AND" globally, but is sufficient to emulate "AND" behavior within this subset.

In the reversible computation model, this type of auxiliary bit is referred to as an ``ancilla bit''. It is initialized to a fixed value prior to computation and serves to preserve information that would otherwise be lost in irreversible operations. A prominent example in quantum computing is the ``Toffoli gate''. This gate operates on three qubits and is defined as follows:

\begin{equation}
    \text{Toffoli} |a, b, c\rangle = |a, b, c \oplus (a \land b)\rangle
\end{equation}

It is evident that this gate is reversible. To realize the AND operation, the auxiliary qubit $c$ is initialized to $0$ prior to computation, so that $c \oplus (a \land b) = a \land b$. Since the NOT operation is also reversible, the OR operation can be simulated by combining it with the AND operation. Consequently, it becomes possible to perform all logical operations equivalent to those achieved by classical gates.

It seems that all operations become reversible. However, in reality, the irreversible component is not eliminated but rather transferred. The initialization of ancilla bits to a specific state results in the loss of their original information, making the process still irreversible. A naive approach would be to directly reset a qubit to the $|0\rangle$ state. However, a significant drawback of this method is the potentially high cost. Furthermore, in complex quantum computations, entanglement may exist among qubits. If the state of a qubit is forcibly altered, it could adversely affect the results of other qubits.

Therefore, a reasonable improvement is to perform the inverse of the original computation. Suppose a computation consists of $n$ steps, represented by:

$$
U = U_1 U_2 \ldots U_n
$$

If the final step, $U_n$, solely modifies the state of the output qubits without affecting the others, and the output qubits aren't changed in other steps, we can then execute the inverse of the preceding steps:

$$
U' = U_{n-1}^{-1} U_{n-2}^{-1} \ldots U_1^{-1}
$$

Consequently, all qubits, with the exception of the output qubits, are restored to their original states. 

Naturally, it is also necessary to reset these output qubits to the $|0\rangle$ state once they are no longer required. When output qubits are intended to serve as inputs for a subsequent process, the previous strategy of reversing the computation can be applied. However, in modern computer architectures, input and output operations involve interaction with the external environment. The preceding discussion relied on the premise that the quantum computation operates within an isolated ``black box''. For instance, if the result is displayed on a screen, this is equivalent to performing a measurement on the qubits. This process is inherently irreversible. Consequently, for a quantum processor to be functional, it must be interfaced with a classical processor to enable information exchange with the real world. At this interface, the state of the qubits collapses. If a reset is required, the physical ``brute force'' methods previously discussed may become necessary. Specific implementation details regarding these techniques are currently a subject of frontier research.

In our group's project, the implementation of Shor's algorithm focused on the underlying logic without strictly ensuring the reversibility of each step. However, following the discussion presented in this article, I constructed an adder using exclusively Toffoli gates and verified its reversibility. It is worth noting that in this specific scenario, the result qubit has a single deterministic state. Consequently, it appears feasible to reverse the entire program.




\section{Something}

Naturally, quantum gates possess capabilities far beyond these examples. The preceding discussion effectively constructed a bijection on $\{0, 1\}^n$, which corresponds to unitary matrices restricted to entries of $0$ or $1$ (i.e., permutation matrices). However, qubits can exist in superpositions, represented as state vectors in a $2^n$-dimensional Hilbert space. Consequently, quantum gates allow for more complex transformations. Thus, a question arises: is it possible to select a finite set of specific quantum gates capable of realizing any arbitrary quantum operation, analogous to how the combination of NOT and AND gates forms a universal set in classical computing?

First take an examination starting with Shor's algorithm. It is known that the algorithm consists of two core components: modular exponentiation and the Quantum Fourier Transform (QFT). While utilizing a single unitary matrix of dimension $2^{O(n)}$ for the computation is conceptually straightforward, physically constructing such a massive quantum gate is infeasible. Consequently, it is necessary to decompose the operation into a product of quantum gates, each acting on a limited number of qubits.

The construction of the QFT transformation matrix was detailed in the previous discussion. The core principle is that the influence of one qubit on another is primarily determined by $2^{\Delta}$, where $\Delta$ denotes the difference in their bit positions, and is independent of the remaining qubits. Specifically, the explicit decomposition can be expressed as follows:

$$
\text{QFT}_n = \text{SWAP}_{\text{all}} \prod_{i=1}^{n} \left( H_i \prod_{j=2}^{n-i+1} CR_{j}^{(i, i+j-1)} \right)
$$

In this expression, $H$ denotes the Hadamard gate, $\text{SWAP}$ denotes the swap gate, and $CR_j$ denotes the controlled rotation gate defined by the following matrices:

$$
CR_j = \begin{bmatrix} 1 & 0 & 0 & 0 \\ 0 & 1 & 0 & 0 \\ 0 & 0 & 1 & 0 \\ 0 & 0 & 0 & e^{2\pi i / 2^j} \end{bmatrix}
$$

For modular exponentiation, the situation is much more complex. The underlying principle begins by expressing the exponent in binary notation, utilizing the method of exponentiation by squaring. Consequently, the power term can be formulated as follows:

$$
c^x = c^{\sum_{i=0}^{n-1} x_i 2^i} = \prod_{i=0}^{n-1} c^{x_i 2^i}
$$

where $x_i \in \{0, 1\}$ represents the $i$-th bit of the exponent $x$.

For the term $c^{2^i} \mod 2^N$, the specific value can be pre-computed using a classical processor. Similarly, this value, denoted here as $y$, can be represented in binary notation. Consequently, the multiplication term can be formulated as follows:

$$
y = c^{2^i} \mod 2^n = \sum_{j=0}^{n-1} y_j 2^j
$$

where $y_j \in \{0, 1\}$ represents the $j$-th bit of $y$.

The remaining operations consist of addition and bit-shifts, which can be implemented using basic quantum gates and several ancilla qubits. However, given the significant implementation complexity, the authors have constructed only a basic adder composed of elementary quantum gates.



Generally, a set of quantum gates capable of simulating any arbitrary quantum operation is termed a ``universal gate set''. It should be noted that there is a fundamental distinction between quantum gates and classic gates: while classical gates operate on discrete binary values ($0$ and $1$), quantum gates operate over the complex field, which is an uncountable set. Consequently, it appears difficult to represent every possible quantum gate using a finite set of elementary gates. However, research indicates that this is achievable if the requirement is relaxed such that sequences of basic gates need only approximate any quantum operation arbitrarily closely. Furthermore, the Solovay-Kitaev theorem demonstrates that such approximation can be performed efficiently for quantum operations acting on a constant number of qubits.


Before addressing this question, it's necessary to first examine the capabilities of quantum gates. It is well known that these gates correspond to unitary matrices. Specifically, a gate acting on $n$ qubits can be represented by a $2^n \times 2^n$ unitary matrix. Together with matrix multiplication, these gates form the unitary group $U(2^n)$.

Although it is widely acknowledged that quantum algorithms can achieve lower time complexity for specific problems compared to their classical counterparts, and that unitary transformations possess significantly greater capabilities, these advantages remain largely theoretical. From the perspective of practical hardware design, the preceding analysis demonstrates that quantum computation requires a substantial number of gates to execute tasks that are simple in the classical domain. Consequently, constraints on the total number of gates become a critical factor. Increasingly, frontier research is shifting focus from merely quantifying theoretical complexity reductions to investigating how quantum systems can outperform classical computers subject to strict gate count limitations.


\section{Citations and Bibliographies}

The use of \BibTeX\ for the preparation and formatting of one's
references is strongly recommended. Authors' names should be complete
--- use full first names (``Donald E. Knuth'') not initials
(``D. E. Knuth'') --- and the salient identifying features of a
reference should be included: title, year, volume, number, pages,
article DOI, etc.

The bibliography is included in your source document with these two
commands, placed just before the \verb|\end{document}| command:
\begin{verbatim}
  \bibliographystyle{ACM-Reference-Format}
  \bibliography{bibfile}
\end{verbatim}
where ``\verb|bibfile|'' is the name, without the ``\verb|.bib|''
suffix, of the \BibTeX\ file.

Citations and references are numbered by default. A small number of
ACM publications have citations and references formatted in the
``author year'' style; for these exceptions, please include this
command in the {\bfseries preamble} (before the command
``\verb|\begin{document}|'') of your \LaTeX\ source:
\begin{verbatim}
  \citestyle{acmauthoryear}
\end{verbatim}


  Some examples.  A paginated journal article \cite{Abril07}, an
  enumerated journal article \cite{Cohen07}, a reference to an entire
  issue \cite{JCohen96}, a monograph (whole book) \cite{Kosiur01}, a
  monograph/whole book in a series (see 2a in spec. document)
  \cite{Harel79}, a divisible-book such as an anthology or compilation
  \cite{Editor00} followed by the same example, however we only output
  the series if the volume number is given \cite{Editor00a} (so
  Editor00a's series should NOT be present since it has no vol. no.),
  a chapter in a divisible book \cite{Spector90}, a chapter in a
  divisible book in a series \cite{Douglass98}, a multi-volume work as
  book \cite{Knuth97}, a couple of articles in a proceedings (of a
  conference, symposium, workshop for example) (paginated proceedings
  article) \cite{Andler79, Hagerup1993}, a proceedings article with
  all possible elements \cite{Smith10}, an example of an enumerated
  proceedings article \cite{VanGundy07}, an informally published work
  \cite{Harel78}, a couple of preprints \cite{Bornmann2019,
    AnzarootPBM14}, a doctoral dissertation \cite{Clarkson85}, a
  master's thesis: \cite{anisi03}, an online document / world wide web
  resource \cite{Thornburg01, Ablamowicz07, Poker06}, a video game
  (Case 1) \cite{Obama08} and (Case 2) \cite{Novak03} and \cite{Lee05}
  and (Case 3) a patent \cite{JoeScientist001}, work accepted for
  publication \cite{rous08}, 'YYYYb'-test for prolific author
  \cite{SaeediMEJ10} and \cite{SaeediJETC10}. Other cites might
  contain 'duplicate' DOI and URLs (some SIAM articles)
  \cite{Kirschmer:2010:AEI:1958016.1958018}. Boris / Barbara Beeton:
  multi-volume works as books \cite{MR781536} and \cite{MR781537}. A
  presentation~\cite{Reiser2014}. An article under
  review~\cite{Baggett2025}. A
  couple of citations with DOIs:
  \cite{2004:ITE:1009386.1010128,Kirschmer:2010:AEI:1958016.1958018}. Online
  citations: \cite{TUGInstmem, Thornburg01, CTANacmart}.
  Artifacts: \cite{R} and \cite{UMassCitations}.

  
\bibliographystyle{ACM-Reference-Format}
\bibliography{sample-base}


%%
%% If your work has an appendix, this is the place to put it.
\appendix

\section{Research Methods}

\subsection{Part One}

Lorem ipsum dolor sit amet, consectetur adipiscing elit. Morbi
malesuada, quam in pulvinar varius, metus nunc fermentum urna, id
sollicitudin purus odio sit amet enim. Aliquam ullamcorper eu ipsum
vel mollis. Curabitur quis dictum nisl. Phasellus vel semper risus, et
lacinia dolor. Integer ultricies commodo sem nec semper.

\subsection{Part Two}

Etiam commodo feugiat nisl pulvinar pellentesque. Etiam auctor sodales
ligula, non varius nibh pulvinar semper. Suspendisse nec lectus non
ipsum convallis congue hendrerit vitae sapien. Donec at laoreet
eros. Vivamus non purus placerat, scelerisque diam eu, cursus
ante. Etiam aliquam tortor auctor efficitur mattis.

\section{Online Resources}

Nam id fermentum dui. Suspendisse sagittis tortor a nulla mollis, in
pulvinar ex pretium. Sed interdum orci quis metus euismod, et sagittis
enim maximus. Vestibulum gravida massa ut felis suscipit
congue. Quisque mattis elit a risus ultrices commodo venenatis eget
dui. Etiam sagittis eleifend elementum.

Nam interdum magna at lectus dignissim, ac dignissim lorem
rhoncus. Maecenas eu arcu ac neque placerat aliquam. Nunc pulvinar
massa et mattis lacinia.

\end{document}
\endinput
%%
%% End of file `sample-sigconf-authordraft.tex'.
